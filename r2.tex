\begin{unit}{Primitive Recursive Functions}

A \strong{computable function} is a function \(f:\N^k\to\N\) for which there is a finite list of simple instructions that can be used to compute the value \(f(m_1,\dots,m_k)\) in a finite number of steps.
These play a very special role in arithmetic as we will soon discover.
While we do have an intuitive understanding of what constitutes ``a finite list of simple instructions,'' we need to make that concept precise in order to be able to understand what a computable function really is.

We have already seen that addition can be described using two simple instructions: \[m + 0 = m \qquad\text{and}\qquad m + n' = (m+n)'.\]
Moreover, multiplication and exponentiation also admit such a description.
Indeed, the Parameterized Recursion Principle is one of the ``simple instructions'' we can use to build computable functions.
More precisely, we will use the following variation as a basic building block for computable functions.

\begin{definition}\label{D:PrimitiveRecursion}
  A function \(h:\N^k\to\N\) is defined by \strong{primitive recursion} using \(f:\N^{k-1}\to\N\) and \(g:\N^{k+1}\to\N\) if \[h(\bar{m},0) = f(\bar{m}) \quad\text{and}\quad h(\bar{m},n+1) = g(\bar{m},n,h(\bar{m},n))\] for all \(n \in \N\) and all parameters \(\bar{m} = (m_1,\dots,m_{k-1}) \in \N^{k-1}.\)
\end{definition}

\noindent
Strictly speaking, this is not exactly a special case of parameterized recursion.
First, we have \(k-1\) parameters instead of just one.
Second, the function \(g\) has an extra input \(n,\) which is not a fixed parameter.
The first problem can be solved by bundling the parameters into one (as suggested by the condensed notation \(\bar{m} = (m_1,\dots,m_{k-1})\)).
Using a similar bundling trick, we see that the function \(\bar{h}(\bar{m},n) = (n,h(\bar{m},n))\) is obtained by parameterized recursion using \(\bar{f}(\bar{m}) = (0,f(\bar{m}))\) and \(\bar{g}(\bar{m},(n,k)) = (n+1,g(\bar{m},n,k)).\)

Combined with a few more basic building blocks, primitive recursion leads to an interesting class of computable functions.

\begin{definition}\label{D:PrimitiveRecursive}
  The \strong{primitive recursive functions} are defined by the following rules.
  \begin{itemize}
  \item The constant zero function \(\nil(n) = 0\) is primitive recursive.
  \item The successor function \(\suc(n) = n+1\) is primitive recursive.
  \item The coordinate projections \(\prj_i(n_1,\dots,n_k) = n_i,\) for \(1 \leq i \leq k,\) are primitive recursive.
  \item If \(f:\N^\ell\to\N\) and \(g_1,\dots,g_\ell:\N^k\to\N\) are primitive recursive functions then so is the composition \(h(n_1,\dots,n_k) = f(g_1(n_1,\dots,n_k),\dots,g_\ell(n_1,\dots,n_k)).\)
  \item If \(f:\N^{k-1}\to\N\) and \(g:\N^{k+1}\to\N\) are primitive recursive function then so is the function \(h:\N^k\to\N\) defined by primitive recursion using \(f\) and \(g\) as in Definition~\ref{D:PrimitiveRecursion}.
  \item No other function is primitive recursive.
  \end{itemize}
\end{definition}

\noindent
Note that the rules are very strict regarding the number of arguments that the functions have.
The natural way to define addition \(h(m,n) = m + n\) is by \[h(m,0) = m \qquad\text{and}\qquad h(m,n+1) = h(m,n)+1,\] but this doesn't exactly fit the pattern for primitive recursion, which requires two primitive recursive functions \(f:\N\to\N\) and \(g:\N^3\to\N\) in order to produce a new primitive recursive function \(h:\N^2\to\N.\)

\begin{problem}
  Show that addition \(h(m,n) = m + n\) is primitive recursive by finding suitable \(f:\N\to\N\) and \(g:\N^3\to\N\) and showing that these are primitive recursive too.
  Then, do the same for multiplication and exponentiation.
\end{problem}

\noindent
\begin{problem}\label{P:PrimitiveRecursiveBase}
Show that the following functions are primitive recursive.
\begin{enumerate}[(a)]
\item \(\displaystyle \pre(n) = \begin{cases} 0 & \text{if $n = 0$,} \\ n-1 & \text{if $n \neq 0$.} \end{cases}\)
\item \(\displaystyle n \sub m = \begin{cases} 0 & \text{if $n < m$,} \\ n-m & \text{if $n \geq m$.} \end{cases}\)
\item \(\displaystyle |n - m| = \begin{cases} m-n & \text{if $n < m$,} \\ n-m & \text{if $n \geq m$.} \end{cases}\)
\item \(\displaystyle \ch(m_0,m_1,n) = \begin{cases} m_0 & \text{if $n = 0$,} \\ m_1 & \text{if $n \neq 0$.} \end{cases}\)
\item \(\displaystyle \max(m,n) = \begin{cases} m & \text{if $n < m$,} \\ n & \text{if $n \geq m$.} \end{cases}\)
\item \(\displaystyle \min(m,n) = \begin{cases} n & \text{if $n < m$,} \\ m & \text{if $n \geq m$.} \end{cases}\)
\end{enumerate}
\end{problem}

\begin{problem}
  Show that if \(f:\N^k\to\N\) is primitive recursive, then so are \[\sum\limits_{i = 0}^n f(m_1,\dots,m_{k-1},i) \quad\text{and}\quad\prod\limits_{i = 0}^n f(m_1,\dots,m_{k-1},i).\]
\end{problem}

There is a fair amount of tedium involved in proving that functions are primitive recursive using Definition~\ref{D:PrimitiveRecursive}.
A typical method for defining functions in mathematics simply spell out how the output is related to the input.
For example, the quotient that results when \(m\) is divided by a positive integer \(n\) can be described as the unique integer \(q\) such that \[nq \leq m \land n(q+1) > m.\]
The resulting function \(q:\N^2\to\N\) (extended with \(q(m,0) = 0,\) say) is primitive recursive but, as you may expect, it is not easy to build this function using primitive recursion.
Fortunately, we can define primitive recursive functions by describing how outputs relate to the inputs, provided that this relationship is simple enough and the function doesn't grow too fast.

\begin{definition}\label{D:PrimitiveRecursivePredicate}
  A property \(\varphi(n_1,\dots,n_k)\) of natural numbers is a \emph{primitive recursive predicate} if \[t_\varphi(n_1,\dots,n_k) = \begin{cases} 1 & \text{if $\varphi(n_1,\dots,n_k)$,} \\ 0 & \text{if $\lnot\varphi(n_1,\dots,n_k)$,} \end{cases}\] is a primitive recursive function.
\end{definition}

\begin{theorem}\label{T:PrimitiveRecursivePredicate}\mbox{}
\begin{enumerate}[\upshape(a)]
\item \(m = n\) and \(m \leq n\) are primitive recursive predicates.
\item If \(\varphi(n_1,\dots,n_k)\) is a primitive recursive predicate and \(f_1,\dots,f_k:\N^\ell\to\N\) are primitive recursive functions then \[\varphi(f_1(m_1,\dots,m_\ell),\dots,f_k(m_1,\dots,m_\ell))\] is also a primitive recursive predicate.
\item If \(\varphi(n_1,\dots,n_k)\) and \(\psi(n_1,\dots,n_k)\) are primitive recursive predicates then so are \[\begin{aligned}
    \varphi(n_1,\dots,n_k) &\land \psi(n_1,\dots,n_k), \\
    \varphi(n_1,\dots,n_k) &\lor \psi(n_1,\dots,n_k), \\
    \varphi(n_1,\dots,n_k) &\lthen \psi(n_1,\dots,n_k), \\
    \varphi(n_1,\dots,n_k) &\liff \psi(n_1,\dots,n_k), \\
  \end{aligned}\] and \[\lnot\varphi(n_1,\dots,n_k).\]
\item
  If \(\varphi(m_1,\dots,m_{k-1},i)\) is a primitive recursive predicate then so are \aside{For bounded quantifiers like these, it is always assumed that \(i\) and \(n\) are different variable symbols, or else the constraint \(i \leq n\) wouldn't be much of a constraint!}
  \[\forall i \leq n\,\varphi(m_1,\dots,m_{k-1},i)\] and
  \[\exists i \leq n\,\varphi(m_1,\dots,m_{k-1},i).\] 
\end{enumerate}
\end{theorem}

\noindent
The next result is a key tool to show that functions are primitive recursive.

\begin{theorem}\label{T:PrimitiveRecursiveDescription}
  A function \(f:\N^k\to\N\) is primitive recursive if \textup(and only if\textup)
  \begin{itemize}
  \item the graph \(n = f(m_1,\dots,m_k)\) is a primitive recursive predicate, and
  \item there is a primitive recursive function \(g:\N^k\to\N\) such that \[f(m_1,\dots,m_k) \leq g(m_1,\dots,m_k)\] for all \(m_1,\dots,m_k \in \N.\)
  \end{itemize}
\end{theorem}

\begin{problem}
  Assuming Theorem~\ref{T:PrimitiveRecursivePredicate} and Theorem~\ref{T:PrimitiveRecursiveDescription}, complete the proof that the quotient function \(q:\N^2\to\N\) is primitive recursive.
  Then show that the remainder function \(r:\N^2\to\N\) is also primitive recursive.
\end{problem}

\begin{problem}
  Prove Theorem~\ref{T:PrimitiveRecursivePredicate}.
  (\textit{Hint}: The primitive recursive functions from Problem~\ref{P:PrimitiveRecursiveBase} may be handy.)
\end{problem}

\begin{problem}
  Suppose \(f:\N^k\to\N\) is a function with a primitive recursive graph, as in Theorem~\ref{T:PrimitiveRecursiveDescription}.
  \begin{enumerate}[(a)]
  \item Show that \[h(m_1,\dots,m_k,i) = \begin{cases} 
      1 & \text{if $i < f(m_1,\dots,m_k)$,} \\
      0 & \text{if $i \geq f(m_1,\dots,m_k)$,}
    \end{cases}\] is primitive recursive.
  \item Show that if \(f(m_1,\dots,m_k) \leq n,\) then \[f(m_1,\dots,m_k) = \sum_{i=0}^n h(m_1,\dots,m_k,i).\]
  \item Prove Theorem~\ref{T:PrimitiveRecursiveDescription}.
  \end{enumerate}
\end{problem}

\begin{problem}\mbox{}
  \begin{enumerate}[(a)]
  \item Show that ``\(n\) is a prime number'' is a primitive recursive predicate.
  \item Show that the prime counting function \[\pi(n) = \text{number of primes $p \leq n$}\] is primitive recursive.
  \item Show that the function \(p:\N\to\N\) that enumerates the prime numbers in increasing order is primitive recursive.
  \end{enumerate}
\end{problem}

\end{unit}
