\begin{unit}{The Concept of Natural Number}

What is a natural number?
This is a difficult question since numbers are abstractions: the number \(3\) is what is common to all collections of three objects.
Early on, we learned to associate \(3\) to such collections and shortly thereafter we learned the arithmetical properties of these numbers.
While this is a perfectly fine working definition of natural number, it is not one that lends itself well to a mathematical analysis.
\aside{The eight solutions are when \(x = \pm1\) and \(y = \pm4,\) and when \(x = \pm7\) and \(y = \pm8.\)}
How can we use such conventions to formally prove that the equation \(15+x^2 = y^2\) has exactly eight integer solutions?

For the sake of concreteness, let's try to fix a definite representation for natural numbers.
To serve as a foundation, this representation should be conceptually simple and it should rely on a minimum of prior notions.
Tally notation satisfies both these criteria.
Let's fix a simple symbol --- say \(\prime\) --- to use as a basic token and define natural numbers as the sequence\aside{The initial comma is not a typo: first element of the sequence is blank!}\[\nat0, \qquad \nat1,  \qquad \nat2, \qquad \nat3, \qquad \nat4, \qquad \ldots\]
This is conceptually simple since each natural number is a collection of that many objects.
We can use these to verify that the equation \[{\underbrace{1+\cdots+1}_{\text{$\nat{15}$ times}}} + x^2 = y^2\] has exactly \(\nat{8}\) integer solutions by exhibiting a one-to-one correspondence between the solutions and the tallies in the list.

Since the generation of our tally lists is reproducible, a concise description of this process suffices to unambiguously define the natural numbers.

\begin{definition}\label{D:N}
  The \strong{standard natural numbers} are defined using the following rules.
  \begin{itemize}
  \item The empty tally list is a natural number.
  \item Appending a tally mark to a natural number results in a natural number.
  \item Nothing else is a natural number.
  \end{itemize}
  The collection of standard natural numbers will be denoted \(\N.\)
\end{definition}

The choice of tally notation in Definition~\ref{D:N} was well motivated but ultimately arbitrary.
The axiomatic method can be used to accommodate alternative choices.

\newcommand{\pax}[1]{\ensuremath{\mathsf{PA}_{\ref{PA:#1}}}}
\begin{definition}\label{D:PA}
  The axiomatic system of \strong{Peano Arithmetic} \textup(\PA\textup) has two primitive symbols:
  \aside{To describe an axiomatic system, one first needs to specify the symbols of the language.
    Logical symbols as well as equality are understood but everything else needs to be specified.
    Once the language is fully specified, one can formulate the axioms of the system.}
  \begin{itemize}
  \item a constant symbol \(0\) \textup (\emph{zero}\textup), and 
  \item a one-place operation symbol \(\mathvisiblespace'\) \textup(\emph{successor}\textup). 
  \end{itemize} 
  The axioms of \PA\ are: 
  \begin{enumerate}[\upshape($\mathsf{PA}_1$)]
  \item\label{PA:succ} \(\forall m, n\,(m = n \liff m' = n'),\)
  \item\label{PA:zero} \(\forall n\,(n' \neq 0),\)
  \item\label{PA:ind} \(\varphi(0) \land \forall n\,(\varphi(n) \lthen \varphi(n')) \lthen \forall n\,\varphi(n)\) for any property \(\varphi.\)
  \end{enumerate}
\end{definition}

\noindent
In the axioms of \PA, the variable symbols \(m\) and \(n\) are intended to range over the domain of natural numbers.
This domain could be anything, so long as it suports a structure that satisfies the axioms of \PA.
The choice of notation \(\mathvisiblespace'\) for the successor operation is meant to suggest the tally mark interpretation of Definition~\ref{D:N}, but the axioms do not prescribe a specific meaning to this choice.
The axioms only prescribe properties that an interpretation must satisfy.

\begin{definition}\label{D:PA}
  A \strong{model of \PA} consists of
  \begin{itemize}
  \item an interpretation of the domain of natural numbers,
  \item an interpretation of \(0\) as an element of the domain,
  \item an interpretation of \(\mathvisiblespace'\) as an operation on elements of the domain,
  \end{itemize}
  in such a way that all axioms of \PA\ are valid using these interpretations.
\end{definition}

Definition~\ref{D:N} provides us with a model of \PA, our \strong{standard model}, where \(0\) is interpreted as the empty tally list and the successor operation is interpreted as appending a tally mark.
\aside{Convince yourself that the standard model satisfies \(\pax{succ}\) and \(\pax{zero}\) before moving on!}
The first two axioms \(\pax{succ}\) and \(\pax{zero}\) are clear.
The third axiom \(\pax{ind}\) is a consequence of the last requirement of Definition~\ref{D:N}.
If \(\varphi\) is a property of our standard natural numbers such that the following are true.
\begin{description}
\item[Base Case ---] The property \(\varphi\) holds of the empty tally list: \(\varphi(0).\)
\item[Successor Step ---] Whenever \(\varphi\) holds of a natural number it also holds of its successor: \(\forall n\,(\varphi(n) \lthen \varphi(n')).\)
\end{description}
Then \(\phi\) must hold of every natural number \(n.\)
Indeed, the natural number \(n\) must be obtained by successively appending tallies starting from empty tally list.
So, we can deduce \(\varphi(n)\) by following the construction of \(n,\) starting from the \emph{Base Case} applying the \emph{Successor Step} each time we append a new tally mark until we finally reach~\(n.\) 
This method, \strong{mathematical induction}, is a very powerful way to establish properties of natural numbers.

The Peano Axioms completely describe the natural numbers in the following sense.

\begin{theorem}[Peano Categoricity Theorem]\label{T:PeanoCategoricity}
  Given any two models \(\N_1\) and \(\N_2\) of \PA,\aside{If we write \(0_1,s_1\) and \(0_2,s_2\) for the interpretations of the constant zero and the successor operation in \(\N_1\) and \(\N_2\) respectively, the requirements of the translation \(h:\N_1 \to \N_2\) can be written symbolically as \(h(0_1) = 0_2\) and \(h(s_1(n)) = s_2(h(n)).\)} there is a \emph{unique} way to translate between elements of the domain of \(\N_1\) and elements of the domain of \(\N_2\) in such a way that the following are true.
  \begin{itemize}
  \item The zero element of \(\N_1\) translates to the zero element of \(\N_2.\)
  \item The successor of an element \(n\) in \(\N_1\) translates to the successor of the translation of \(n\) in \(\N_2.\)
  \end{itemize}
\end{theorem}

\noindent
Such a translation is called an \strong{isomorphism}.
In general, isomorphic structures are structurally the same but the implementation details may be different.

An immediate consequence of Theorem~\ref{T:PeanoCategoricity} is that every model of \PA\ is isomorphic to our standard model \(\N.\) 
In other words, any other model of \PA\ is indistinguishable from our standard model except for inessential details such as how numbers are actually represented.
For computational purposes, decimal and binary representation is much more efficient than tally notation.
The only advantage of tally notation is that it provides a very simple representation of zero and successor, the only two aspects of natural numbers that appear in \PA.

In fact, Theorem~\ref{T:PeanoCategoricity} says more: the isomorphism between any two models of \PA\ is unique.
So we never need to worry about how to translate between a model of \PA\ and our standard model \(\N\) since there is only one possible translation!


\begin{problem}
  Peano described addition using two simple rules:
  \begin{itemize}
  \item \(m + 0 = m\)
  \item \(m + n' = (m + n)'\)
  \end{itemize}
  \begin{enumerate}[(a)]
  \item Use these rules to show that \(2 + 2 = 4,\) or rather \(\tal2 + \tal2 = \tal4.\)
  \item Explain how these two rules can be used to correctly compute the sum of any two standard natural numbers.
  \end{enumerate}
\end{problem}

\begin{problem}
  Peano described multiplication using two simple rules:
  \begin{itemize}
  \item \(m \times 0 = 0\)
  \item \(m \times n' = m \times n + m\)
  \end{itemize}
  \begin{enumerate}[(a)]
  \item Use these rules to show that \(2 \times 2 = 4,\) or rather \(\tal2 \times \tal2 = \tal4.\)
  \item Explain how these two rules can be used to correctly compute the product of any two standard natural numbers.
  \end{enumerate}
\end{problem}

\begin{problem}
  Design similar computation rules of exponentiation and use them to show that \(2^2 = 4.\) 
\end{problem}

\begin{theorem}[Recursion Principle]\label{T:Recursion}
  Given an object \(x_0 \in A\) and a function \(f:A \to A\) there is a \emph{unique} function \(h:\N \to A\) such that \[h(0) = x_0 \quad\text{and}\quad \forall n\,(h(n') = f(h(n))).\]
\end{theorem}

\begin{problem}
  The following steps outline a proof of the Recursion Principle.
  Let \(x_0 \in A,\) \(f:A \to A\) and so on be as in the statement of the theorem.
  A finite list \(u_0,u_1,\dots,u_n\) of elements of \(A\) is an \emph{attempt at \(h\)} if \(u_0 = x_0\) and any two successive list elements are related by \(u_{i'} = f(u_i).\)
  \begin{enumerate}[(a)]
  \item Show that if \(u_0,u_1,\dots,u_n\) and \(v_0,v_1,\dots,v_m\) are two attempts at \(h\) then we must have \(u_i = v_i\) for \(0 \leq i \leq m,n.\)
  \item Show that for every \(n \in \N\) there is a (necessarily unique) list \(u_0,u_1,\dots,u_n\) of length \(n'\) which is an attempt at \(h.\)
  \item Show that the function \(h:\N \to X\) defined by \(h(n) = u_n,\) where \(u_0,u_1,\dots,u_n\) is the unique attempt at \(h\) with length \(n',\) satisfies the conclusion of the theorem.
  \end{enumerate}
\end{problem}

\begin{theorem}[Parameterized Recursion Principle]\label{T:ParameterizedRecursion}
  Given a parameter set \(P\) and two functions \(x:P \to A\) and \(f:P \times A \to A,\) there is a \emph{unique} function \(h:P \times \N \to A\) such that for every parameter \(p \in P,\) we have \[h(p,0) = x(p) \quad\text{and}\quad \forall n\,(h(p,n') = f(p,h(n))).\] 
\end{theorem}

\begin{problem}
  Prove the Parameterized Recursion Principle.
\end{problem}

\begin{problem}
  Prove the Peano Categoricity Theorem.
\end{problem}

\begin{problem}
  The rules of arithmetic are all consequences of \PA\ and the above definitions of addition, multiplication and exponentiation.
  To convince yourself of this, prove the following familiar facts about addition:
  \[\begin{aligned}
    (\ell + m) + n &= \ell + (m+n), & 0 + n &= n, & m + n &= n + m.
  \end{aligned}\]
\end{problem}

\end{unit}
\endinput
