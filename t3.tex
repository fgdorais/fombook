\begin{unit}{Inductive Types}

The prototypical example of an inductive type is the type of natural numbers.
Before we introduce that idea, let's take a step back and think about natural numbers in the category of sets.
Is it possible to describe the natural numbers using a universal property similar to those for products, coproducts and exponentials?

As we saw in Chapter~I, Peano Arithmetic has two key components: zero and successor.
The successor is a distinguished function \(\Succ:\N\to\N,\) which makes perfect sense in the categorical context.
However, zero is a distinguished element of \(\N,\) which refers to the set-theoretic notion of membership.
To make sense of this notion in the categorical context, we view elements of a set \(A\) as functions \(1 \to A,\) where the ordinal \(1\) is the prototypical set with one element.
That way, we can view zero as a distinguished function \(\Zero:1\to \N\) which points to the zero element of \(\N.\)
Using this trick, we can reformulate the Recursion Principle (Chapter~I, Theorem~\ref{T:Recursion}) in the category of sets.

\begin{theorem}[Universal Property of Natural Numbers]\label{T:UPN}
  Given \(x_0:1 \to A\) and \(f:A \to A,\) there is a unique \(h:\N\to A\) such that \(h \circ \Zero = x_0\) and \(h \circ \Succ = f \circ h.\)
  \aside{Make sure you understand how Theorem~\ref{T:UPN} relates to the Recursion Theorem before moving on!}
  \[\xymatrix@R1em{
    & \N \ar@{.>}[dd]|-{h} & \N \ar@{.>}[dd]|-{h} \ar[l]_-{\Succ} \\
    1 \ar[ur]^-{\Zero} \ar[dr]_-{x_0}  \\
    & A & A \ar[l]^-{f}
  }\]
\end{theorem}

\begin{problem}
  Explain how Theorem~\ref{T:UPN} is related to the Recursion Principle.
\end{problem}

\begin{problem}
  Formulate the Peano Categoricity Theorem (Chapter~I, Theorem~\ref{T:PeanoCategoricity}) in the category of sets.
  Then prove it using the universal property of Theorem~\ref{T:UPN}.
\end{problem}

Our goal is to formulate the type of natural numbers in the context of Type Theory.
So let's try to formulate rules that correspond to the universal property of Theorem~\ref{T:UPN}.
To get started, we need a formation rule that merely declares the existence of the type of natural numbers:
\aside{This is indeed a rule, it is a rule that has no hypotheses!}
\[\ded{}{\type{\N}} \tag{$\N$-Form.}\]
To populate the type \(\N,\) we have two introduction rules for zero and the successors.
We no longer have to think of zero as a function \(1 \to \N\) since we can simply declare zero to be an object of type \(\N.\)
\[\frac{}{0 : \N} \qquad \frac{t:\N}{t':\N} \tag{$\N$-Intro.}\]
From these we successively derive the sequence of natural numbers \[0, \quad 1 \defn 0', \quad 2 \defn 1', \quad 3 \defn 2', \quad\ldots\] which populate the type \(\N.\)

\begin{problem}
  Derive a term for the successor function \(\Succ:\N\to\N.\)
\end{problem}

We now turn to the Recursion Principle, or rather the universal property of Theorem~\ref{T:UPN}.
The universal property tells us exactly what rules we need for the type of natural numbers.
\[\ded{r : A \to A \quad s : A}{\elim{\N}[r,s]:\N \to A} \tag{$\N$-Rec.}\]
So, if we have a type \(A\) equipped with a distinguished function \(r: A \to A\) and a distinguished element \(s:A,\) then we have \(\elim{\N}[r,s]:\N \to A,\) which is intended to be the \(h:\N \to A\) from Theorem~\ref{T:UPN}.
The requirements on \(h:\N \to A\) are captured by two computation rules:
\[\begin{gathered}
  \frac{r : A \to A \qquad s : A}{\elim{\N}[r,s](0) \defn s} \\[\jot]
  \frac{r : A \to A \qquad s : A \qquad t : \N}{\elim{\N}[r,s](t') \defn r(\elim{\N}[r,s](t))}
\end{gathered}\tag{$\N$-Comp.}\]

\begin{problem}
  Explain how every primitive recursive function can be faithfully interpreted in Type Theory. 
\end{problem}

The general pattern for inductive types is as follows:
\begin{itemize}
\item A formation rule to introduce the type.
\item Some introduction rules that allow us to construct elements of that type.
\item A recursion rule that allows us to construct functions from that type to similar types.
\item Some computation rules that explain how the function obtained from the recursion rule applies to elements obtained from the introduction rules. 
\end{itemize}
One of the simplest inductive type is the \strong{unit type}, the type with exactly one element: 
\[\ded{}{\type{1}} \tag{$1$-Form.}\]
\[\ded{}{\star : 1} \tag{$1$-Intro.}\]
\[\ded{t:A}{\elim1[t]:1 \to A} \tag{$1$-Rec.}\]
\[\ded{t:A}{\elim1[t](\star) \defn t} \tag{$1$-Comp.}\]
In terms of sets, the canonical interpretation of \(1\) is simply the ordinal \(1 = \set0\) and \(\star\) is its unique element, but any other singleton set would work just as well to interpret the unit type.

\begin{problem}
  What is the universal property of \(1\) in the context of the category of sets?
\end{problem}

\begin{problem}
  \mbox{}\aside{This is often called the \strong{boolean type} and its elements are often called `true' and `false'.}%
  Formulate rules similar to those of the unity type for the inductive type with exactly two elements.
\end{problem}

\begin{problem}
  The unit type is simple because its introduction rules have no hypotheses.
  The \strong{coproduct type} is a simple example of an inductive type where the introduction rules have hypotheses.
  \[\ded{\type{A} \quad \type{B}}{\type{A + B}} \tag{$+$-Form.}\]
  \[\ded{s : A}{\inl(s) : A + B} \qquad \ded{t : B}{\inr(t) : A + B} \tag{$+$-Intro.}\]
  Complete the definition of coproduct types with recursion and computation rules that match the universal property of Theorem~\ref{T:UPS}.
\end{problem}

\begin{problem}
  Product types are actually inductive types. What is the recursion rule for product types? 
\end{problem}

Another important example of inductive types are \strong{list types}.
\[\ded{\type{A}}{\type{\List{A}}} \tag{$\List{}$-Form.}\]
This type is intended to capture the idea of a finite list of elements of type \(A.\)
It has two introduction rules corresponding to the empty list, which is denoted \(\Null,\) and another for appending one element to an existing list:
\[\ded{}{\Null : \List{A}} \qquad \ded{t : A \quad u : \List{A}}{\Cons(t,u) : \List{A}} \tag{$\List{}$-Intro.}\]
The recursion and computation rules for lists is similar to that for natural numbers:
\[\ded{r:A \times B \to B \quad s:B}{\elim{\List{A}}[r,s]:\List{A}\to B} \tag{$\List{}$-Rec.}\]
\[\begin{gathered}
\ded{r:A \times B \to B \quad s:B}{\elim{\List{A}}[r,s](\Null) \defn s} \\  
\ded{r:A \times B \to B \quad s:B \quad t : A \quad u:\List{A}}{\elim{\List{A}}[r,s](\Cons(t,u)) \defn r(t,\elim{\List{A}}[r,s](u))}
\end{gathered} \tag{$\List{}$-Comp.}\]

\begin{problem}
  An important operation on list is \emph{concatenation}: putting one list at the end of another.
  Formulate a term \(\operatorname{cat}:\List{A}\times\List{A} \to \List{A}\) that represents concatenation.
\end{problem}

\begin{problem}\mbox{}   
  \begin{enumerate}[(a)]
  \item Interpret list types in the category of sets and formulate their universal propery by analogy with Theorem~\ref{T:UPN}.
  \item Using the universal property from part~(a) and that of Theorem~\ref{T:UPN} to show that \(\List{1}\) is isomorphic to \(\N\) in the category of sets.
  \item Using the rules for natural numbers and list types, derive terms for the functions \(\List{1} \to \N\) and \(\N \to \List{1}\) you found in part~(b).
  \end{enumerate}
\end{problem}

\end{unit}
\endinput
